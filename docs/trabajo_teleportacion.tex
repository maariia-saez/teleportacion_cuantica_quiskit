\documentclass[11pt, a4paper]{article}

\usepackage[utf8]{inputenc}
\usepackage[T1]{fontenc}
\usepackage[spanish]{babel}
\usepackage[margin=2.5cm]{geometry} 
\usepackage{amsmath}
\usepackage{amssymb}
\usepackage{graphicx}
\usepackage{parskip} 
\usepackage{csquotes}
\usepackage{tensor}
\usepackage{booktabs}
\usepackage{caption}
\usepackage{float} 
\usepackage{enumitem} 

% --- PAQUETE PARA ENCABEZADOS Y PIES DE PÁGINA ---
\usepackage{fancyhdr}
\pagestyle{fancy}
\fancyhf{} 
\fancyfoot[L]{\footnotesize \textit{Análisis de la Teleportación Cuántica}} 
\fancyfoot[R]{\thepage} 
\renewcommand{\headrulewidth}{0pt} 
\renewcommand{\footrulewidth}{0.4pt} 

% Enlaces
\usepackage{hyperref}
\hypersetup{
    colorlinks=true,
    linkcolor=black,
    filecolor=magenta,
    urlcolor=cyan,
    citecolor=red,
    pdftitle={Análisis de Teleportación Cuántica},
    pdfauthor={Alicia Elvira Montes Núñez y María Sáez Díaz}
}

% --- CÓDIGO PYTHON ---
\usepackage{xcolor}      
\usepackage{listings}    

\definecolor{codegreen}{rgb}{0,0.6,0}  
\definecolor{codeblue}{rgb}{0,0,0.6}   
\definecolor{codered}{rgb}{0.6,0,0}    
\definecolor{backcolour}{rgb}{0.98,0.98,0.98} 

\lstdefinestyle{mystyle}{
    language=Python,
    backgroundcolor=\color{backcolour},    
    commentstyle=\color{codegreen},
    keywordstyle=\color{codeblue},
    stringstyle=\color{codered},
    basicstyle=\ttfamily\footnotesize, 
    breakatwhitespace=false,          
    breaklines=true,                 
    captionpos=b,                    
    keepspaces=true,                 
    numbers=left,                    
    numbersep=5pt,                   
    showspaces=false,                
    showstringspaces=false,
    showtabs=false,                  
    tabsize=2,
    frame=single,                    
    rulecolor=\color{black},
    literate={á}{{\'a}}1 {é}{{\'e}}1 {í}{{\'i}}1 {ó}{{\'o}}1 {ú}{{\'u}}1 {ñ}{{\~n}}1
}
\lstset{style=mystyle}

\newcommand{\ket}[1]{|#1\rangle}
\newcommand{\bra}[1]{\langle#1|}

\begin{document}

% --- PORTADA PERSONALIZADA ---
\begin{titlepage}
    \centering
    \vspace*{1cm}
    
    {\Huge \textbf{Análisis Teórico y Experimental de la Teleportación Cuántica}} \\
    \vspace{0.5cm}
    {\Large \textit{Del Teorema de No-Clonación a la Internet Cuántica}} \\
    \vspace{1.5cm}
    
    % --- IMAGEN PORTADA ---
    % Asegúrate de tener la imagen subida en Overleaf
    \includegraphics[width=0.8\textwidth]{imagenes/portada.png} 
    
    \vspace{4cm}
    
    \textbf{\Large Autoras:} \\
    \vspace{0.2cm}
    {\Large Alicia Elvira Montes Núñez} \\
    {\Large María Sáez Díaz}
    
    \vfill
    
    {\large \textbf{Introducción a la Información y Computación Cuántica}} \\
    \vspace{0.5cm}
    
\end{titlepage}

% --- ÍNDICE ---
\thispagestyle{empty} 
\tableofcontents
\newpage

% --- INICIO DEL CONTENIDO (Página 1) ---
\setcounter{page}{1} 

\begin{abstract}
Este trabajo analiza el protocolo de teleportación cuántica, desde su concepción teórica como solución a la imposibilidad de clonar estados cuánticos, hasta su primera e histórica demostración experimental. Se estudia el problema fundamental del ``Teorema de No-Clonación'', se describe en detalle el protocolo teórico de Bennett et al. (BBCJPW) desarrollando la formalización matemática completa del cambio de base, y se detalla la ingeniosa implementación experimental de Zeilinger et al. en 1997. Se explica cómo se contrastó la teoría y qué limitaciones técnicas se enfrentaron. Finalmente, se revisa la evolución de esta técnica y se presentan simulaciones detalladas en Qiskit, analizando la ejecución tanto en simuladores ideales como en hardware cuántico real (\texttt{ibm\_fez}), discutiendo las fuentes de error y la fidelidad obtenida.
\end{abstract}

% SECCIÓN 1
\section{Introducción}

\subsection{El Problema Fundamental: La Información Cuántica}
La transmisión de información es un proceso directo y robusto en el mundo clásico. Un flujo de bits puede ser leído, copiado y retransmitido millones de veces sin degradación significativa. Sin embargo, en el dominio cuántico, la información se comporta de manera radicalmente distinta debido a dos principios rectores: el principio de superposición y el colapso de la función de onda.

Un estado cuántico puro, representado por un vector de estado $\ket{\psi}$, encapsula toda la información de una partícula. Matemáticamente, un qubit en superposición se describe como una combinación lineal de los estados base:
\begin{equation}
    \ket{\psi} = \alpha\ket{0} + \beta\ket{1}
\end{equation}
Donde $\alpha$ y $\beta$ son números complejos (amplitudes de probabilidad) que cumplen la condición de normalización $|\alpha|^2 + |\beta|^2 = 1$. Estos coeficientes contienen la información cuántica.

La intuición clásica de ``medir y copiar'' para transmitir esta información no es aplicable por dos razones fundamentales:
\begin{enumerate}
    \item \textbf{Colapso de la función de onda:} El acto de medir el estado $\ket{\psi}$ fuerza al sistema a colapsar irreversiblemente en uno de sus estados base, $\ket{0}$ o $\ket{1}$, con probabilidad $|\alpha|^2$ y $|\beta|^2$ respectivamente. Este proceso destruye la superposición original, impidiendo conocer los coeficientes originales $\alpha$ y $\beta$. Al medir, perdemos la información cuántica.
    \item \textbf{Teorema de No-Clonación:} Incluso si se evita una medición directa, existe una prohibición estructural en la mecánica cuántica que impide duplicar información desconocida.
\end{enumerate}

\subsection{El Teorema de No-Clonación}
Demostrado formalmente por Wootters y Zurek en 1982 \cite{wootters1982}, este teorema establece que la linealidad de la mecánica cuántica prohíbe la existencia de un operador unitario universal $U_{clon}$ que pueda generar una copia idéntica de un estado arbitrario desconocido.

La demostración por reducción al absurdo es ilustrativa y fundamental para entender la necesidad de la teleportación. Supongamos que existe un clonador tal que, para los estados base ortogonales, funciona correctamente copiándolos sobre un estado auxiliar $\ket{b}$:
$$ U_{clon}(\ket{0}\otimes\ket{b}) = \ket{0}\otimes\ket{0} $$
$$ U_{clon}(\ket{1}\otimes\ket{b}) = \ket{1}\otimes\ket{1} $$
Ahora, consideremos un estado en superposición general $\ket{\psi} = \alpha\ket{0} + \beta\ket{1}$. Debido a la \textbf{linealidad} de los operadores cuánticos, la acción del clonador sobre este estado debe ser la suma de las acciones sobre sus partes:
\begin{equation}
    U_{clon}(\ket{\psi}\otimes\ket{b}) = \alpha U_{clon}(\ket{0}\otimes\ket{b}) + \beta U_{clon}(\ket{1}\otimes\ket{b}) = \alpha\ket{00} + \beta\ket{11}
\end{equation}
Sin embargo, el resultado que esperamos de una verdadera clonación es el producto tensorial de $\ket{\psi}$ consigo mismo:
\begin{equation}
    \ket{\psi}\otimes\ket{\psi} = (\alpha\ket{0} + \beta\ket{1}) \otimes (\alpha\ket{0} + \beta\ket{1}) 
\end{equation}
Expandiendo este producto:
\begin{equation}
    = \alpha^2\ket{00} + \alpha\beta\ket{01} + \beta\alpha\ket{10} + \beta^2\ket{11}
\end{equation}
Al comparar ambas ecuaciones, es evidente la contradicción. La versión lineal carece de los términos cruzados ($\ket{01}$ y $\ket{10}$) que son esenciales en la copia verdadera. Esta contradicción demuestra que la clonación perfecta es imposible para un estado general \cite{zeilinger2000}.

\subsection{Justificación y Objetivos}
La solución teórica a este callejón sin salida fue propuesta en 1993 por Bennett et al. (BBCJPW) \cite{bbcjpw1993}: la \textbf{teleportación cuántica}. Este protocolo no viola el teorema de no-clonación porque el estado original se destruye en el proceso de envío. Utiliza el entrelazamiento como un ``canal'' para desensamblar la información cuántica en una parte clásica (2 bits) y una correlación cuántica, reconstruyéndola en el destino.

El objetivo de este trabajo es analizar la evolución de este protocolo:
\begin{enumerate}
    \item Desarrollar la fundamentación matemática completa del protocolo ideal, explicitando los cambios de base.
    \item Analizar la primera implementación experimental de 1997 y sus desafíos fotónicos.
    \item Comparar la teleportación probabilística con la determinista (iones).
    \item Implementar simulaciones en Qiskit detalladas para visualizar el impacto del ruido real y analizar la fidelidad del hardware cuántico actual.
\end{enumerate}

% SECCIÓN 2
\section{Análisis Teórico: El Protocolo Ideal (BBCJPW)}
El protocolo requiere tres qubits: el qubit 1 (el estado $\ket{\psi}$ que Alice quiere enviar) y un par de qubits entrelazados (2 y 3) que forman el recurso cuántico compartido. Alice tiene los qubits 1 y 2, mientras que Bob tiene el qubit 3.

\subsection{Desarrollo Matemático Completo}
Supongamos que el par entrelazado compartido es el singlete de Bell $\ket{\Psi^-}_{23} = \frac{1}{\sqrt{2}}(\ket{0}_2\ket{1}_3 - \ket{1}_2\ket{0}_3)$. El estado inicial de todo el sistema de tres partículas es el producto tensorial del estado de Alice y el par entrelazado:

\begin{align}
    \ket{\Psi}_{total} &= \ket{\psi}_1 \otimes \ket{\Psi^-}_{23} \nonumber \\
    &= (\alpha\ket{0}_1 + \beta\ket{1}_1) \otimes \frac{1}{\sqrt{2}}(\ket{0}_2\ket{1}_3 - \ket{1}_2\ket{0}_3) \nonumber \\
    &= \frac{1}{\sqrt{2}} (\alpha\ket{0}_1\ket{0}_2\ket{1}_3 - \alpha\ket{0}_1\ket{1}_2\ket{0}_3 + \beta\ket{1}_1\ket{0}_2\ket{1}_3 - \beta\ket{1}_1\ket{1}_2\ket{0}_3)
\end{align}

Esta ecuación describe el sistema en la base computacional estándar ($\ket{00}, \ket{01}...$). Sin embargo, Alice va a realizar una medición conjunta en la base de Bell sobre los qubits 1 y 2. Para saber qué pasará, debemos reescribir algebraicamente los pares $\ket{0}_1\ket{0}_2$, $\ket{0}_1\ket{1}_2$, etc., en términos de los estados de Bell.

Recordamos la base de Bell:
\begin{align*}
    \ket{\Phi^\pm} &= \frac{1}{\sqrt{2}}(\ket{00} \pm \ket{11}) \\
    \ket{\Psi^\pm} &= \frac{1}{\sqrt{2}}(\ket{01} \pm \ket{10})
\end{align*}
Invirtiendo estas relaciones, podemos expresar la base computacional en función de la base de Bell:
\begin{align*}
    \ket{00} &= \frac{1}{\sqrt{2}}(\ket{\Phi^+} + \ket{\Phi^-}) & \ket{11} &= \frac{1}{\sqrt{2}}(\ket{\Phi^+} - \ket{\Phi^-}) \\
    \ket{01} &= \frac{1}{\sqrt{2}}(\ket{\Psi^+} + \ket{\Psi^-}) & \ket{10} &= \frac{1}{\sqrt{2}}(\ket{\Psi^+} - \ket{\Psi^-})
\end{align*}

Sustituimos estas identidades en la ecuación del estado total $\ket{\Psi}_{total}$. Este es el paso clave donde se mezcla la información de Alice con el entrelazamiento. Al agrupar los términos por el estado de Bell de los qubits 1 y 2, obtenemos:

\begin{align}
    \ket{\Psi}_{total} = \frac{1}{2} [ & \ket{\Psi^-}_{12} \otimes (-\alpha\ket{0}_3 - \beta\ket{1}_3) \nonumber \\
                                       & + \ket{\Psi^+}_{12} \otimes (-\alpha\ket{0}_3 + \beta\ket{1}_3) \nonumber \\
                                       & + \ket{\Phi^-}_{12} \otimes (\alpha\ket{1}_3 + \beta\ket{0}_3) \nonumber \\
                                       & + \ket{\Phi^+}_{12} \otimes (\alpha\ket{1}_3 - \beta\ket{0}_3) ]
\end{align}

Esta ecuación es fundamental. Revela que inmediatamente después de la medición de Alice, el qubit de Bob colapsa instantáneamente en uno de cuatro estados posibles. Cada resultado tiene una probabilidad de $p = |1/2|^2 = 1/4$ (25\%). Nótese que el estado de Bob contiene los coeficientes $\alpha$ y $\beta$, pero ''desordenados'' o con fases cambiadas.

\subsection{El Canal Clásico y la Reconstrucción Unitaria}
Para recuperar el estado original $\ket{\psi}$, Alice debe comunicar el resultado de su medición (2 bits clásicos) a Bob. Bob aplica entonces una transformación unitaria (puerta lógica cuántica) para corregir el estado.

Es importante aclarar que este proceso no entra en conflicto con la \textbf{Relatividad Especial}. Aunque el cambio en el estado cuántico ocurre al instante, Bob no puede ``leer'' la información por sí solo: para él, su qubit parece comportarse de forma totalmente aleatoria, como si fuera ruido, hasta que Alice le dice qué corrección aplicar. Como Alice debe enviar esos 2 bits necesarios por un canal clásico (que sí respeta el límite de la velocidad de la luz), la información útil nunca viaja más rápido de lo permitido.

\begin{itemize}
    \item \textbf{Si Alice mide $\ket{\Psi^-}_{12}$ (bits 00):} El estado de Bob es $-(\alpha\ket{0} + \beta\ket{1})$. Salvo una fase global irrelevante (-1), es el estado original. Bob aplica la puerta \textit{Identidad ($I$)}.
    \item \textbf{Si Alice mide $\ket{\Psi^+}_{12}$ (bits 01):} El estado es $-\alpha\ket{0} + \beta\ket{1}$. El término $\beta$ tiene el signo correcto, pero $\alpha$ no (fase relativa). Bob aplica la puerta \textit{Pauli-Z ($Z$)} para corregir la fase.
    \item \textbf{Si Alice mide $\ket{\Phi^-}_{12}$ (bits 10):} El estado es $\alpha\ket{1} + \beta\ket{0}$. Los bits están invertidos (flip). Bob aplica la puerta \textit{Pauli-X ($X$)} para intercambiar 0 y 1.
    \item \textbf{Si Alice mide $\ket{\Phi^+}_{12}$ (bits 11):} El estado es $\alpha\ket{1} - \beta\ket{0}$. Hay inversión de bits y de fase. Bob aplica primero \textit{$X$ y luego $Z$}.
\end{itemize}

% SECCIÓN 3
\section{Análisis Experimental: La Primera Demostración (1997)}
En 1997, el grupo de Anton Zeilinger en la Universidad de Innsbruck logró la primera demostración experimental de la teleportación cuántica utilizando fotones polarizados \cite{bouwmeester1997}.

\subsection{Configuración Experimental y Fuente SPDC}
El experimento utilizó la polarización de los fotones como qubit físico ($\ket{H} \equiv \ket{0}, \ket{V} \equiv \ket{1}$).
La generación de pares entrelazados se logró mediante el proceso de \textbf{Conversión Paramétrica Descendente Espontánea (SPDC) de Tipo-II}. 

En este proceso no lineal, un haz láser ultravioleta pulsado atraviesa un cristal BBO (beta-barium borate). Por conservación de energía y momento, un fotón ultravioleta del bombeo se divide ocasionalmente en dos fotones infrarrojos con polarizaciones ortogonales, creando el estado entrelazado $\ket{\Psi^-}$.
Para crear el qubit 1 (el estado a teleportar), se retro-reflejaba el haz de bombeo para que pasara una segunda vez por el cristal, generando otro par independiente.

\subsection{El Desafío de la BSM Fotónica: Efecto Hong-Ou-Mandel}
El mayor obstáculo técnico era realizar la Medida de Estado de Bell sobre dos fotones independientes. En óptica lineal, una BSM completa es imposible porque requiere una interacción fuerte entre fotones. La solución fue explotar la interferencia cuántica mediante el \textit{Efecto Hong-Ou-Mandel (HOM)} en un divisor de haz 50:50.
\begin{itemize}
    \item Si los fotones 1 y 2 inciden en un estado simétrico ($\ket{\Psi^+}, \ket{\Phi^\pm}$), interfieren constructivamente y salen juntos por el mismo puerto (\textit{bunching}).
    \item Si inciden en el estado antisimétrico ($\ket{\Psi^-}$), la interferencia destructiva fuerza a los fotones a separarse (\textit{anti-bunching}).
\end{itemize}
Por lo tanto, registrar una coincidencia simultánea en detectores opuestos es la firma física del estado $\ket{\Psi^-}$.

\subsection{Resultados y Limitaciones}
Debido a que el montaje solo podía identificar 1 de los 4 estados de Bell, el protocolo solo tenía éxito cuando la naturaleza elegía aleatoriamente ese resultado (25\% de eficiencia). A pesar de ello, se confirmó que el estado de polarización se transfería con una fidelidad superior al límite clásico.

% SECCIÓN 4
\section{Evolución y Experimentos Posteriores}
\begin{itemize}
    \item \textbf{Entanglement Swapping (1998):} Pan et al. demostraron la teleportación del propio entrelazamiento \cite{pan1998}, un paso clave para los repetidores cuánticos.
    \item \textbf{Teleportación Determinista (2004):} Grupos en NIST e Innsbruck usaron iones atrapados \cite{riebe2004}. Al poder realizar una BSM completa con iones, lograron un éxito del 100\% (determinista).
    \item \textbf{Larga Distancia:} Desde los 143 km en las Islas Canarias \cite{pan2012_143km} hasta la teleportación Tierra-Satélite (Micius) a más de 500 km \cite{yin2017_satellite}.
\end{itemize}

% SECCIÓN 5
\section{Implementación y Análisis con Qiskit}
Una parte fundamental de este trabajo ha sido la validación empírica mediante el SDK de IBM Quantum (Qiskit). Hemos diseñado tres experimentos computacionales para contrastar la teoría ideal con la realidad del hardware NISQ (Noisy Intermediate-Scale Quantum).

\subsection{Simulación 1: Protocolo Ideal y Estrategia de Verificación}
El objetivo de esta simulación es verificar la lógica matemática del protocolo asumiendo compuertas perfectas.
Para verificar que el estado ha sido teleportado correctamente sin "mirarlo" (lo cual colapsaría el estado), utilizamos una técnica de inversión unitaria:
\begin{enumerate}
    \item Preparamos el estado $\ket{\psi} = R_y(\theta)\ket{0}$ en el qubit de Alice.
    \item Ejecutamos el protocolo de teleportación completo, incluyendo las correcciones condicionales.
    \item En el qubit de Bob, aplicamos la rotación inversa $R_y(-\theta)$.
\end{enumerate}
Si el estado de Bob era realmente $\ket{\psi}$, al aplicar la inversa, el estado debe regresar a $\ket{0}$. Por tanto, \textbf{medir '0' significa éxito}.

\begin{lstlisting}[language=Python, caption=Código Qiskit con Verificación Inversa]
# 1. Preparación del estado en q0
qc.ry(theta, 0)
# 2. Entrelazamiento (q1, q2)
qc.h(1)
qc.cx(1, 2)
# 3. Medida de Bell
qc.cx(0, 1)
qc.h(0)
qc.measure([0, 1], [0, 1])
# 4. Correcciones en q2 (Bob)
with qc.if_test((cr, 1)):  qc.z(2) # Si bit0=1 -> Z
with qc.if_test((cr, 2)):  qc.x(2) # Si bit1=1 -> X
# 5. Verificación: Invertimos la preparación
qc.ry(-theta, 2)
qc.measure(2, 2)
\end{lstlisting}

\begin{figure}[H]
    \centering
    \begin{minipage}{0.55\textwidth}
        \centering
        \includegraphics[width=\linewidth]{imagenes/Simulacion1Circuito.png}
        \captionof{figure}{Circuito Ideal con correcciones.}
    \end{minipage}%
    \hfill
    \begin{minipage}{0.40\textwidth}
        \centering
        \includegraphics[width=\linewidth]{imagenes/Simulacion1Histograma.png}
        \captionof{figure}{Fidelidad del 100\%.}
    \end{minipage}
    \caption*{Figura 1: El histograma muestra que Bob siempre mide '0' (la barra azul). Esto confirma que las correcciones funcionan perfectamente: independientemente de lo que mida Alice (eje X: 00, 01, 10, 11), Bob recupera el estado.}
\end{figure}

\subsection{Simulación 2: Réplica Probabilística (1997)}
Aquí simulamos la limitación del equipo de Zeilinger eliminando el paso 4 (correcciones). Esto es equivalente a no tener el canal clásico activo para modificar el estado de Bob en tiempo real.

\begin{lstlisting}[language=Python, caption=Código sin correcciones (Probabilístico)]
# ... Preparación y BSM idénticas ...
# OMITIMOS los bloques if_test()
# Bob no hace nada (Aplica Identidad)
qc.ry(-theta, 2) # Verificación directa
qc.measure(2, 2)
\end{lstlisting}

\begin{figure}[H]
    \centering
    \begin{minipage}{0.55\textwidth}
        \centering
        \includegraphics[width=\linewidth]{imagenes/Simulacion2Circuito.png}
        \captionof{figure}{Circuito sin feedback clásico.}
    \end{minipage}%
    \hfill
    \begin{minipage}{0.40\textwidth}
        \centering
        \includegraphics[width=\linewidth]{imagenes/Simuacion2Histograma.png}
        \captionof{figure}{Resultados dispersos.}
    \end{minipage}
    \caption*{Figura 4: Interpretación de datos. La teleportación exitosa (Bit de Bob = 0) solo ocurre sistemáticamente cuando Alice mide '11' (correspondiente a $\ket{\Psi^-}$). En los otros tres casos, obtenemos errores, confirmando la eficiencia del 25\% reportada en 1997.}
\end{figure}

\subsection{Simulación 3: Ejecución en Hardware Real (ibm\_fez)}
Para la prueba final, conectamos con el ordenador cuántico superconductor \texttt{ibm\_fez} usando \texttt{QiskitRuntimeService}.
El circuito lógico ideal tuvo que ser \textit{transpilado}: el compilador de Qiskit adaptó nuestras puertas lógicas (H, CNOT) a las puertas físicas nativas del procesador y a su topología de conectividad (qué qubits están conectados con cuáles).

\textbf{Análisis de Resultados:}
Ejecutamos 4096 disparos (\textit{shots}). A diferencia del simulador ideal, el histograma muestra una pequeña barra roja en el estado '1' para el qubit de Bob.

\begin{figure}[H]
    \centering
    \begin{minipage}{0.55\textwidth}
        \centering
        \includegraphics[width=\linewidth]{imagenes/Simulacion3Circuito.png}
        \captionof{figure}{Circuito transpilado para IBM.}
    \end{minipage}%
    \hfill
    \begin{minipage}{0.40\textwidth}
        \centering
        \includegraphics[width=\linewidth]{imagenes/Simulacion3Histograma.png}
        \captionof{figure}{Ruido en hardware real.}
    \end{minipage}
    \caption*{Figura 6: Resultados reales. Obtuvimos 3862 éxitos y 234 fallos.}
\end{figure}

La \textit{Fidelidad (F)} del proceso se calcula como la fracción de éxitos:
\[ F = \frac{N_{0}}{N_{total}} = \frac{3862}{4096} \approx 94.3\% \]

Este 5.7\% de error no se debe a fallos en el protocolo, sino a limitaciones físicas del dispositivo NISQ:
\begin{itemize}
    \item \textbf{Errores de puerta (CNOT):} Las puertas de dos qubits son ruidosas y introducen errores de fase.
    \item \textbf{Decoherencia ($T_1, T_2$):} Los qubits pierden su información cuántica antes de que termine el circuito.
    \item \textbf{Errores de lectura (SPAM):} El propio aparato de medición tiene una pequeña probabilidad de leer '1' cuando el estado es '0'.
\end{itemize}

% SECCIÓN 6
\section{Conclusión y Trabajo Futuro}

\subsection{Conclusión}
La teleportación cuántica ha transitado de ser una paradoja teórica a una herramienta de ingeniería esencial. El protocolo BBCJPW resolvió elegantemente el conflicto con el Teorema de No-Clonación, utilizando el entrelazamiento como recurso consumible. Las simulaciones realizadas confirman que, mientras la teoría es perfecta, la implementación física ha evolucionado desde soluciones probabilísticas hasta sistemas deterministas, aunque el ruido en los procesadores actuales (como hemos visto en \texttt{ibm\_fez}) sigue siendo un desafío para la escalabilidad.

\subsection{Aplicaciones Futuras: Criptografía y Seguridad Cuántica}
Los principios fundamentales analizados en este trabajo —la superposición, el entrelazamiento y, crucialmente, el Teorema de No-Clonación— constituyen la base teórica de la Criptografía Cuántica. Mientras que en la teleportación el no-clonado exige la destrucción del estado original para su transmisión, en la \textbf{Distribución de Claves Cuánticas (QKD)} este mismo teorema garantiza la seguridad incondicional de la comunicación.

Protocolos como el \textbf{BB84} explotan la imposibilidad de medir un estado cuántico sin perturbarlo (colapso de la función de onda). En un escenario práctico, si una entidad externa (Eva) intentase interceptar los fotones transmitidos entre Alice y Bob para copiar la información, la física cuántica asegura que su intervención introducirá errores estadísticos detectables (QBER). 
Por tanto, la próxima frontera en la aplicación de estos conceptos no es solo mover información, sino asegurarla, utilizando la fidelidad de la transmisión como métrica para certificar la ausencia de espías en el canal.

% BIBLIOGRAFÍA EN PÁGINA APARTE
\newpage
\begin{thebibliography}{99} 

\bibitem{wootters1982}
W. K. Wootters and W. H. Zurek,
\textit{A single quantum cannot be cloned},
Nature, vol. 299, pp. 802–803, 1982.

\bibitem{wootters2009}
W. K. Wootters and W. H. Zurek,
\textit{The no-cloning theorem},
Physics Today, vol. 62, no. 2, pp. 76–77, 2009.

\bibitem{bbcjpw1993}
C. H. Bennett, G. Brassard, C. Crépeau, R. Jozsa, A. Peres, and W. K. Wootters,
\textit{Teleporting an Unknown Quantum State via Dual Classical and Einstein-Podolsky-Rosen Channels},
Physical Review Letters, vol. 70, no. 13, pp. 1895–1899, 1993.

\bibitem{zeilinger2000}
A. Zeilinger,
\textit{Quantum Teleportation},
Scientific American, vol. 282, no. 4, pp. 50–59, 2000.

\bibitem{bouwmeester1997}
D. Bouwmeester, J. W. Pan, K. Mattle, M. Eibl, H. Weinfurter, and A. Zeilinger,
\textit{Experimental quantum teleportation},
Nature, vol. 390, pp. 575–579, 1997.

\bibitem{pan1998}
J. W. Pan, D. Bouwmeester, H. Weinfurter, and A. Zeilinger,
\textit{Experimental entanglement swapping: entangling photons that never interacted},
Physical Review Letters, vol. 80, no. 18, pp. 3891–3894, 1998.

\bibitem{riebe2004}
M. Riebe, H. Häffner, C. F. Roos, W. Hänsel, J. Benhelm, G. P. T. Lancaster, T. W. Körber, C. Becher, F. Schmidt-Kaler, D. F. V. James, and R. Blatt,
\textit{Deterministic quantum teleportation with atoms},
Nature, vol. 429, pp. 734–737, 2004.

\bibitem{pan2012_143km}
X. S. Ma, T. Herbst, T. Scheidl, D. Wang, S. Kropatschek, W. Naylor, B. Wittmann, A. Mech, J. Kofler, E. Anisimova, V. Makarov, T. Jennewein, R. Ursin, and A. Zeilinger,
\textit{Quantum teleportation over 143 kilometres using active feed-forward},
Nature, vol. 489, pp. 269–273, 2012.

\bibitem{yin2017_satellite}
J. Yin, Y. Cao, Y. H. Li, S. K. Liao, L. Zhang, J. G. Ren, W. Q. Cai, W. Y. Liu, B. Li, H. Dai, et al.,
\textit{Ground-to-satellite quantum teleportation},
Nature, vol. 549, pp. 70–73, 2017.

\bibitem{github_repo}
A. E. Montes Núñez and M. Sáez Díaz,
\textit{Repositorio de simulación de teleportación cuántica con Qiskit},
GitHub, 2025. Disponible: \url{https://github.com/maariia-saez/teleportacion_cuantica_qiskit}

\end{thebibliography}

\end{document}